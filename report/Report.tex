\documentclass{article}
\usepackage{multicol}				%per usare multicols {2}
\title{Fake News Analysis}			% titolo iniziale

\begin{document}
	\pagenumbering{arabic}			% numeri di pagina
	\maketitle						% inserisce titolo \title{}}
	
    
    \newpage						% nuova pag
    \begin{multicols}{2}			% divide pagina in 2 colonne
    	\section{Introduzione}
	    	In questo documento verra' affrontata la tematica dell'individuazione delle Fake News, con particolare attenzione rivolta all'analisi lessicale e di come sia possibile discriminare le notizie vere da quelle false basandosi su tecniche di Natural Language Processing.
	    	
	    	Il documento presenta, nella prima parte, un'analisi esplorativa delle "features": caratteristiche estraibili dalle notizie che possono essere sfruttate per discriminare quelle vere da quelle false. Ne sono presentate diverse e, per ciascuna, viene evidenziato se si tratta di una caratteristica discriminante o meno.
	    	Nella seconda parte, alcune di queste features vengono utilizzate per rilevare le Fake News, in particolare basandosi su caratteristiche intrinseche nel testo e nelle parole che compongono la notizia.
	    	
	    	Una volta isolate alcune Fake News, si procede ad analizzarne manualmente il testo per cercare ulteriori indicatori o eventuali anomalie presenti.
	    \section{s2}
	    \subsection{ss2}
		    ssssss
		    ssssss
		    
   	\end{multicols}

\end{document}